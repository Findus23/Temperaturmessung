\documentclass[12pt,a5paper,landscape,DIV=20]{scrartcl}
\usepackage[utf8]{inputenc}
\usepackage[ngerman]{babel}
\usepackage[T1]{fontenc}
\usepackage[colorlinks, linkcolor = black, citecolor = black, filecolor = black, urlcolor = blue]{hyperref} 
\author{Lukas Winkler}

\let\raggedsection\centering

\begin{document}
\vspace*{\fill}
\section*{Informationen}

Weiterführende Informationen gibt es unter folgenden Quellen:
\begin{itemize}
\item \href{http://winkler.kremszeile.at/}{winkler.kremszeile.at}: Live-Kopie der Webseite des Raspberry Pi
\item VWA: meine Vorwissenschaftliche Arbeit mit allen Details zum Projekt:\newline \href{https://github.com/Findus23/VWA/raw/master/main.pdf}{github.com/Findus23/VWA/raw/master/main.pdf}
\item Github: \href{https://github.com/Findus23/Umweltdatenmessung}{github.com/Findus23/Umweltdatenmessung}
	\begin{itemize}
	\item Der komplette Programmcode und alle anderen Dateien, die zum Projekt gehören, sind hier gesammelt.
	\item Auch alle Veränderungen seit Dezember 2013 können hier angesehen werden:  \url{https://github.com/Findus23/Umweltdatenmessung/commits/master}
	\item Große Veränderungen werden zusätzlich separat gelistet: \url{https://github.com/Findus23/Umweltdatenmessung/releases}
	\end{itemize}
\item Flickr: \href{https://www.flickr.com/photos/findus23/sets/72157637721138445/}{www.flickr.com/photos/findus23/sets/72157637721138445}
	\begin{itemize}
	\item Hier sind über 100 Bilder vom Projekt zu sehen.
	\end{itemize}
\item Youtube: \url{https://www.youtube.com/playlist?list=PLjtFdocVknd4aw90_zVr0U4BFlRH9PatA}
	\begin{itemize}
	\item Einige Videos (z.B. vom Display) sind auf Youtube zu finden.
	\end{itemize}
\end{itemize}
\vspace*{\fill}
\end{document}